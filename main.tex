%-----------------------------------------------------------------------------------------------------------
% Everything before \begin{document} and
% \end{document} is called Preamble. The Preamble
% sets up your document. The preamble typically specifies
%   the document class
%   languages
%   imports packages by using \usepackage{package_name} command
%-----------------------------------------------------------------------------------------------------------

%-----------------------------------------------------------------------------------------------------------
% Note: A latex command consist of a backslash, command name, argument (with {}) #
% and optional arguments (with []), e.g.
% \command_name[optonal_argument1, optional_argument2 ..]{arugment}
%-----------------------------------------------------------------------------------------------------------

%-----------------------------------------------------------------------------------------------------------
% \documentclass have to be the first command of each latex file and 
% defines layout and class of the document.
%   [options]: contains options to define the layout of the document (fontsize, paper, columns ...) 
%   {class}: defines the class of the document. Possible classes are:
%               - article: for scientifc journals
%               - book: for books
%               - reports: for longer reports with several chapters, thesis, small books
%               - letter: for letters
%               - slides: for slides
%               - beamer: for latex presentations
%   Sources: 
%       https://texblog.org/2013/02/13/latex-documentclass-options-illustrated/ (for options)
%       https://en.wikibooks.org/wiki/LaTeX/Document_Structure#Document_classes (for classes)
%-----------------------------------------------------------------------------------------------------------
\documentclass[a4paper,11pt,onecolumn]{report}

%-----------------------------------------------------------------------------------------------------------
% The \usepackage{package_name} commands imports external packages (here: graphicx)
% to provide latex with external features.
%-----------------------------------------------------------------------------------------------------------
% graphicx package enables to import external graphic files
\usepackage{graphicx}

%-----------------------------------------------------------------------------------------------------------
% For adding a title, author and date, latex need to include first the following three commands
% in preamble. In a second you have to include the \maketitle command between \begin{docuemtn} and
% \end{document}
%-----------------------------------------------------------------------------------------------------------
% adds a title
\title{Soccer is great \& Politics sucks!}   
% adds a author with footnote to thank, e.g. your institution or coworker
\author{Sefa Kutlu!\thanks{Thanks to Albert Einstein whose shared his great knowledge with us. }}
% adds a date (\today for todays date or a apsicifc date, e.g. \date{August 2022})
\date{\today}                           

%-----------------------------------------------------------------------------------------------------------
% The content of a document have to be added between the \begin{document} and \end{document}
%-----------------------------------------------------------------------------------------------------------
\begin{document}
    % \maketitle commands adds the information defined above with title, author date commands to the document
    \maketitle
    Hello Latex!. This is an example sentence written by me.
    % \\ new line command
    \\
    % bold, italic, underline commands
    % \textbf{word}: The word in braces will displayed bold
    The last word should be \textbf{bold}.      
    \\
    % \textit{word}: The word in braces will displayed italic
    The last word should be \textit{italic}.
    \\
    % \underline{word}: The word will be underlined
    The last word should be \underline{underlined}.
    \\
    % combine of the three commands from above
    The last three words should be \textbf{\textit{\underline{bold italic underlined}}}.


\end{document}